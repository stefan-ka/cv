% This work may be distributed and/or modified under the
% conditions of the LaTeX Project Public License version 1.3c,
% available at http://www.latex-project.org/lppl/.


\documentclass[11pt,a4paper,sans]{moderncv}        % possible options include font size ('10pt', '11pt' and '12pt'), paper size ('a4paper', 'letterpaper', 'a5paper', 'legalpaper', 'executivepaper' and 'landscape') and font family ('sans' and 'roman')

% moderncv themes
\moderncvstyle{classic}                             % style options are 'casual' (default), 'classic', 'banking', 'oldstyle' and 'fancy'
\moderncvcolor{blue}                               % color options 'black', 'blue' (default), 'burgundy', 'green', 'grey', 'orange', 'purple' and 'red'
%\renewcommand{\familydefault}{\sfdefault}         % to set the default font; use '\sfdefault' for the default sans serif font, '\rmdefault' for the default roman one, or any tex font name
\nopagenumbers{}                                   % uncomment to suppress automatic page numbering for CVs longer than one page

% character encoding
\usepackage[utf8]{inputenc}                       % if you are not using xelatex ou lualatex, replace by the encoding you are using

% adjust the page margins
\usepackage[scale=0.75]{geometry}
\setlength{\hintscolumnwidth}{3.2cm}                % if you want to change the width of the column with the dates
%\setlength{\makecvtitlenamewidth}{10cm}           % for the 'classic' style, if you want to force the width allocated to your name and avoid line breaks. be careful though, the length is normally calculated to avoid any overlap with your personal info; use this at your own typographical risks...

% personal data
\name{Stefan}{Kapferer}
\title{Software Architect, University Lecturer in Software Engineering and Researcher}
\address{Dioggstrasse 3}{8640 Rapperswil SG}{Switzerland}
% \phone[mobile]{}
\email{stefan@kapferer.ch}
\homepage{stefan.kapferer.ch}
\social[linkedin]{stefankapferer}
\social[github]{stefan-ka}
%\extrainfo{additional information}
\photo[160pt][0.0pt]{ska}                       % optional, remove / comment the line if not wanted; '64pt' is the height the picture must be resized to, 0.4pt is the thickness of the frame around it (put it to 0pt for no frame) and 'picture' is the name of the picture file

% bibliography adjustements (only useful if you make citations in your resume, or print a list of publications using BibTeX)
%   to show numerical labels in the bibliography (default is to show no labels)
\makeatletter\renewcommand*{\bibliographyitemlabel}{\@biblabel{\arabic{enumiv}}}\makeatother
%   to redefine the bibliography heading string ("Publications")
%\renewcommand{\refname}{Articles}


%----------------------------------------------------------------------------------
%            content
%----------------------------------------------------------------------------------
\begin{document}
%-----       resume       ---------------------------------------------------------
\makecvtitle

\section{Experience}
\cventry{06/2023 -- Present}{Lecturer in Software Engineering and Researcher}{Institute for Software at Eastern Switzerland University of Applied Sciences (OST)}{Rapperswil SG}{}{
\begin{itemize}%
\item Lecturing Object-Oriented Programming, Software Engineering, Cloud Solutions.
\item Supervision of study projects and bachelor theses.
\item Research projects in the area of Software Engineering, Domain-Driven Design (for example Context Mapper), Ethical and Value-Based Software Engineering.
\item Consulting, coaching and trainings in the following topics: Software Architecture, Domain-Driven Design (DDD) and/or Context Mapper, Cloud-Native Applications
\end{itemize}}

\cventry{01/2021 -- 05/2023}{Software Architect}{mimacom ag}{Zurich}{}{Architect and tech-lead in software engineering projects, DDD expert, leader of a team of six software engineers.
\begin{itemize}%
\item Software Architect role in several in-house projects and in consulting assignments at customer site.
\item Planning, design and implementation of software components and integration solutions.
\item Leading project teams regarding technology, architecture and design (professional \& technical coaching).
\item Domain-Driven Design (DDD) consulting (in-house and in customer projects)
	\begin{itemize}
		\item Advice and support software engineers in implementing tactical DDD.
		\item Advice/consult on modeling (domain models) and the use of Context Mapper.
		\item Consulting in projects on strategic domain-driven design (service decomposition).
		\item Active participation and participation in the architecture guild, DDD guild and requirements engineering guild (contribution of know-how in relation to DDD).
		\item Writing articles and blow posts on the subject of DDD.
	\end{itemize}
\item Leading a team of six Software Engineers within the company.
	\begin{itemize}
		\item Professional and technical coaching of employees.
		\item Support of the onboarding process for new employees.
		\item Conducting regular employee appraisals.
		\item Conducting development and feedback meetings with employees.
		\item Advising employees on personal development, further education and training.
		\item Development of employees within the company (level \& salary).
	\end{itemize}
\end{itemize}}

\cventry{01/2021 -- 05/2023}{Software Architect}{mimacom ag (continued)}{}{}
{\begin{itemize}%
\item Technologies and methods:
	\begin{itemize}
		\item Onion architecture and Domain-Driven Design (DDD)
		\item Microservice architectures
		\item RESTful HTTP, Open API specifications (Swagger), messaging with RabbitMQ
		\item Arc42 software documentation, Context Mapper, PlantUML
		\item Spring Framework
		\item Java 8-17, Kotlin
		\item Angular, Typescript
		\item Docker, Kubernetes, OpenShift
		\item GIT, Maven, Gradle, GitLab
		\item JPA, Hibernate, Flyway
		\item PostgreSQL, Oracle DB
		\item JUnit, TDD, ArchUnit
	\end{itemize}
\end{itemize}}

\cventry{02/2020 -- 12/2020}{Software Engineer and Project Lead}{Institute for Software (IFS) at \newline University of Applied Sciences of Eastern Switzerland (OST / HSR)}{\newline Rapperswil SG}{}{Leading development and research around the Context Mapper project. Supporting other research projects and teaching on software architecture, (micro-)service-oriented architectures, and Domain-driven Design (DDD). \newline{}%
Involved projects and activities:%
\begin{itemize}%
\item Context Mapper (\href{https://contextmapper.org}{contextmapper.org}) open source project, \newline ``A Modeling Framework for Strategic Domain-driven Design''
	\begin{itemize}
		\item Leading project technically
		\item Writing research papers on Context Mapper and Domain-driven Design related topics
	\end{itemize}
\item Microservice DSL (MDSL) project \newline (\href{https://microservice-api-patterns.github.io/MDSL-Specification/}{microservice-api-patterns.github.io/MDSL-Specification})
  \begin{itemize}%
    \item Supporting Prof. Olaf Zimmermann in implementing MDSL generator tooling
    \item Generating Java code (``moduliths''), GraphQL schemas, protocol buffer specifications, and Open API specifications out of MDSL contracts
      \begin{itemize}
  	    \item Conceptual work: Map abstract API contracts to concrete interface technologies. Derive service contracts from DDD-based architecture models.
      \end{itemize}
  \end{itemize}
\item Teaching support in application architecture on B.Sc. level
	\begin{itemize}
		\item Domain-driven Design (DDD)
		\item Context Mapper
	\end{itemize}
\end{itemize}}


\cventry{05/2018 -- 09/2018}{Software Engineering (Intern)}{Ergon Informatik AG}{Zurich}{}{\begin{itemize}
	\item Implementation of business web application in the real estate industry
	\item Implementation and maintenance of build and testing platforms
	\item Conducting customer trainings and a requirements engineering workshop
\end{itemize}
Used technologies and frameworks: HTML, CSS, JS, Freemarker, Java, Magnolia
CMS, Gradle, Maven, Ansible \newline{}}

\cventry{01/2014 -- 08/2017 (part-time: 60\%)}{Software Engineer Professional}{Adcubum AG}{St. Gallen}{}{\begin{itemize}
	\item Design and implementation of tools to support developers (Eclipse plugins, command line tools, etc.)
	\item Development and realization of concepts concerning CI/CD
	\item Development, operation, and maintenance of ’adcubum SYRIUS’ build and it’s continuous integration environment with Gradle and Jenkins
	\item Development, integration, and operation of software documentation tools;
	\item Classic software engineering tasks with Java and J2EE
\end{itemize}}

\cventry{01/2014 -- 08/2017 (part-time: 60\%)}{Software Engineer Professional}{Adcubum AG (continued)}{}{}{\begin{itemize}
	\item Documentation of architecture and design
	\item Supporting the migration from SVN to GIT
	\item Supporting developers regarding IDE (Eclipse), Build (Gradle), SCM (Git), Con-
tinuous Integration, etc.
\end{itemize}}

\cventry{10/2012 -- 12/2013}{Software Engineer}{Adcubum AG}{St. Gallen}{}{Product development (adcubum SYRIUS):
\begin{itemize}
	\item Development of new features and enhancements in the module \newline ``Bestandes-/Vertragsverwaltung''
	\item Creation of SQL scripts for database migration (Oracle DB)
	\item Architecture optimization for business processes within the module
	\item Supporting software quality assurance processes
\end{itemize}
Used systems and technologies: Java, JEE, Eclipse, SVN, Oracle SQL, XML, Linux}

\cventry{05/2011 -- 09/2012}{Software Developer Java}{swoffice AG}{Teufen}{}{\begin{itemize}
	\item Product development (CRM solution)
	\item Testing and operating solution (cloud-based solution)
	\item Collaboration with customers in multiple projects
\end{itemize}
Used technologies: Java, JavaFX, Spring Framework, Apache Lucene, PostgreSQL}

\cventry{09/2008 -- 04/2011}{Software Developer}{clavis IT ag}{Herisau}{}{\begin{itemize}
	\item Development of solution concepts
	\item Specification and design of solutions in direct collaboration with customers
	\item Implementation of solutions based on the following technologies, frameworks, and products: IBM Lotus Notes/Domino, Liferay Portal, Java/JEE (Tomcat and WebSphere Express), Web technologies (CSS, JS, HTML, XML), Webservices
	\item Testing and documentation
	\item Supporting customers with operating implemented solutions
\end{itemize}}

\cventry{08/2004 -- 08/2008}{Apprenticeship in informatics (application development)}{clavis IT ag}{Herisau}{}{\begin{itemize}
	\item Software development (IBM Lotus Notes, Java, web technologies)
\end{itemize}}

\section{Education}
\cventry{09/2018 -- 01/2020}{M.Sc. in Engineering - Focusing on Information and Communication Technologies in the Software and Systems Master Research Unit}{University of Applied Sciences of Eastern Switzerland (HSR FHO)}{Rapperswil}{}{Conducted several research projects on  service decomposition, (micro-)service-oriented architectures, and Domain-driven Design (DDD) aiming to strengthen personal knowledge in software architecture. The open source tool \href{https://contextmapper.org/}{Context Mapper} is a result of these projects.}

\cventry{09/2017 -- 04/2018}{Completed 21 credits towards M.Sc. in Computer Science}{ETH Zurich}{Zurich}{}{Completed the following courses: Algorithms and Data Structures, Theoretical Computer Science, Linear Algebra}

\cventry{09/2013 -- 08/2017 (part-time)}{B.Sc. in Computer Science}{University of Applied Sciences of Eastern Switzerland (HSR FHO)}{Rapperswil}{}{\textit{Thesis:} Developed a concept and prototypic implementation for an architectural refactoring of the data access security based on attribute-based access control (ABAC) in a standard software for the insurance sector.}

\cventry{10/2009 -- 10/2012}{Advanced Federal Diploma of Higher Education (HF) in Computer Science}{Zentrum für berufliche Weiterbildung (ZbW)}{St. Gallen}{}{\textit{Thesis:} Concept for a ``State of the art Java development environment with Continuous Integration (CI)'' and implementation in a case study project.}

\cventry{08/2004 -- 08/2008}{Apprenticeship in informatics (application development)}{clavis IT ag}{Herisau}{}{Federal Diploma of Vocational Education and Training (EFZ)}

\section{Theses, papers, and publications}

\cventry{2024}{\href{https://doi.org/10.1109/MS.2023.3322312}{Continuous Integration and Delivery in Open Source Development and Pattern Publishing: Lessons Learned With Tool Setup and Pipeline Evolution}}{Olaf Zimmermann, Cesare Pautasso, Stefan Kapferer, Mirko Stocker}{IEEE Software}{IEEE, volume 41, issue 1}{\href{https://doi.org/10.1109/MS.2023.3322312}{https://doi.org/10.1109/MS.2023.3322312}}

\cventry{2022}{\href{https://blog.mimacom.com/ddd-and-context-mapper-experience/}{``Domain-Driven Design (DDD) in Practice -- Experience with Context Mapper''}}{Stefan Kapferer}{Blogpost}{}{\href{https://blog.mimacom.com/ddd-and-context-mapper-experience/}{blog.mimacom.com/ddd-and-context-mapper-experience}}

\cventry{2021}{\href{https://contextmapper.org/media/SD-00-Java-06-SP-Kapferer-Zimmermann.pdf}{``Domain-Driven Design in der Praxis - Erfahrungen mit dem Open-Source-Tool Context Mapper'' (GERMAN)}}{Stefan Kapferer, Olaf Zimmermann}{JavaSPEKTRUM}{6/2021, pages 20-23}{\href{https://contextmapper.org/background-and-publications/}{https://contextmapper.org/background-and-publications/}}

\cventry{2021}{\href{https://doi.org/10.1007/978-3-030-67445-8\_11}{Domain-driven Architecture Modeling and Rapid Prototyping with Context Mapper (Revised and Extended Selected Papers)}}{Stefan Kapferer, Olaf Zimmermann}{8th International Conference on Model-Driven Engineering and Software Development - MODELSWARD}{Springer CCIS, volume 1361}{\href{https://doi.org/10.1007/978-3-030-67445-8\_11}{https://doi.org/10.1007/978-3-030-67445-8\_11}}

\cventry{2020}{Domain-driven Service Design - Context Modeling, Model Refactoring and Contract Generation}{Stefan Kapferer, Olaf Zimmermann}{Papers From the 14th Advanced Summer School on Service-Oriented Computing (SummerSOC'20)}{}{\href{https://doi.org/10.1007/978-3-030-64846-6\_11}{https://doi.org/10.1007/978-3-030-64846-6\_11}}

\cventry{2020}{\href{https://doi.org/10.5220/0008910502990306}{Domain-specific Language and Tools for Strategic Domain-driven Design, Context Mapping and Bounded Context Modeling}}{Stefan Kapferer, Olaf Zimmermann}{Proceedings of the 8th International Conference on Model-Driven Engineering and Software Development - MODELSWARD, pages 299-306}{INSTICC, SciTePress}{\href{https://doi.org/10.5220/0008910502990306}{https://doi.org/10.5220/0008910502990306}}

\cventry{2020}{A Modeling Framework for Strategic Domain-driven Design and Service Decomposition}{Stefan Kapferer}{Master thesis}{University of Applied Sciences of Eastern Switzerland (HSR FHO)}{Proposing a modular architecture for a Strategic Domain-driven Design (DDD) modeling framework including a DSL, a reverse engineering framework, Architectural Re-
factorings (ARs), service decomposition analysis on the basis of coupling criteria,
and multiple graphical generators. More information can be found in the open source project \href{https://contextmapper.org}{contextmapper.org} for more information. \href{https://eprints.ost.ch/821/}{https://eprints.ost.ch/821/}}

\cventry{2019}{Service Decomposition as a Series of Architectural Refactorings}{Stefan Kapferer}{Term project}{University of Applied Sciences of Eastern Switzerland (HSR FHO)}{\href{https://eprints.ost.ch/784/}{https://eprints.ost.ch/784/}}

\cventry{2019}{Empirical Research in Software Engineering}{Stefan Kapferer}{Seminar paper}{University of Applied Sciences of Eastern Switzerland (HSR FHO)}{Scientific research strategies applied to software engineering. \newline \href{https://eprints.ost.ch/820/}{https://eprints.ost.ch/820/}}

\cventry{2018}{A Domain-specific Language for Service Decomposition}{Stefan Kapferer}{Term project}{University of Applied Sciences of Eastern Switzerland (HSR FHO)}{A formal approach to strategic Domain-driven Design (DDD) and service decomposition implemented in the Context Mapper open source tool. \newline \href{https://eprints.ost.ch/722/}{https://eprints.ost.ch/722/}}

\cventry{2018}{Model Transformations for DSL Processing}{Stefan Kapferer}{Seminar paper}{University of Applied Sciences of Eastern Switzerland (HSR FHO)}{Proof of concept for the implementation of refactorings for Domain-specific Languages (DSLs) on the basis of the Henshin tool and algebraic graph transformations. \newline \href{https://eprints.ost.ch/819/}{https://eprints.ost.ch/819/}}

\cventry{2017}{``Attributbasierte Autorisierung in einer Branchenlösung für das Versicherungswesen''}{Stefan Kapferer, Samuel Jost}{Bachelor thesis}{University of Applied Sciences of Eastern Switzerland (HSR FHO)}{Concept and prototypic implementation for attribute-based access control (ABAC) in a (micro-)service-oriented architecture. \href{https://eprints.ost.ch/602/}{https://eprints.ost.ch/602/}}

\cventry{2016}{``Architectural Refactoring der Data Access Security''}{Stefan Kapferer}{Semester thesis}{University of Applied Sciences of Eastern Switzerland (HSR FHO)}{Extracting the data access security of a monolithic application into a separate
Bounded Context (DDD) or microservice. \href{https://eprints.ost.ch/564/}{https://eprints.ost.ch/564/}}

\section{Talks and Presentations}

\cventry{2023}{Domain-Driven Design (DDD) with Context Mapper}{1-Day Workshop}{CH Open Workshoptage 2023}{}{}

\cventry{2022}{Strategic Domain-driven Design Advanced and Context Mapper}{Application architecture course}{University of Applied Sciences of Eastern Switzerland (OST)}{}{}

\cventry{2020}{Strategic Domain-driven Design Advanced and Context Mapper}{Application architecture course}{University of Applied Sciences of Eastern Switzerland (OST)}{}{}


\cventry{2020}{Domain-driven Service Design - Context Modeling, Model Refactoring and Contract Generation}{Conference talk}{14th Symposium and Summer School On Service-Oriented Computing}{}{\href{https://contextmapper.org/media/Stefan-Kapferer\_SummerSoC2020\_presentation.pdf}{Slides Link}}

\cventry{2019}{Context Mapper: DSL and Tools for Domain-driven Service Design - Bounded context modeling and microservice decomposition}{Java meetup talk}{Java User Group CH}{}{\href{https://www.jug.ch/html/events/2019/context_mapper.html}{Slides Link}}

\section{Certifications \& Awards}
\cventry{2023}{''Starter Kit in Hochschuldidaktik`` at Eastern Switzerland University of Applied Sciences (OST)}{}{}{}{}
\cventry{2022}{iSAQB Advanced Module: IMPROVE - ''Evolution und Verbesserung von Softwarearchitekturen``}{}{}{}{}
\cventry{2021}{iSAQB Certified Professional for Software Architecture - Foundation Level}{}{}{}{}
\cventry{2020}{SummerSoC Young Researcher Award 2020}{\newline For the paper: ``Domain-driven Service Design - Context Modeling, Model Refactoring and Contract Generation''}{\newline Symposium and Summer School on Service-oriented Computing}{}{Sponsored by \href{https://servtech.info/}{ServTech} (Scientific Academy for Service Technology)}

\section{Open Source Projects}
\cvitemwithcomment{}{Open source projects I contribute(d) to:}{}

\cventry{}{Context Mapper: \newline A Modeling Framework for Strategic Domain-driven Design}{}{}{}{\href{https://github.com/ContextMapper}{github.com/ContextMapper}}

\cventry{}{Microservice Domain-Specific Language (MDSL)}{}{}{}{\href{https://github.com/Microservice-API-Patterns/MDSL-Specification}{github.com/Microservice-API-Patterns/MDSL-Specification}}

\cventry{}{Lakeside Mutual}{a fictitious insurance company which serves as a sample application to demonstrate microservices and domain-driven design (DDD)}{}{}{\href{https://github.com/Microservice-API-Patterns/LakesideMutual}{github.com/Microservice-API-Patterns/LakesideMutual}}

\cventry{}{Software / Service / API Design Practice Repository (DPR)}{}{}{}{\href{https://github.com/socadk/design-practice-repository}{github.com/socadk/design-practice-repository}}

\cventry{}{protobufgen: A Java Library to Serialize Protocol Buffer Files}{}{}{}{\href{https://github.com/Microservice-API-Patterns/protobufgen}{github.com/Microservice-API-Patterns/protobufgen}}

\section{Languages}
\cvitemwithcomment{German}{Mother tongue}{}
\cvitemwithcomment{English}{Fluent}{Certificate in Advanced English (University of Cambridge)}

\section{Interests}
\cvlistitem{Software Engineering}
\cvlistitem{Software Architecture and Design}
\cvlistitem{Domain-driven Design (DDD)}
\cvlistitem{Object-oriented Analysis and Design (OOAD)}

% Publications from a BibTeX file without multibib
%  for numerical labels: \renewcommand{\bibliographyitemlabel}{\@biblabel{\arabic{enumiv}}}% CONSIDER MERGING WITH PREAMBLE PART
%  to redefine the heading string ("Publications"): \renewcommand{\refname}{Articles}
\nocite{*}
\bibliographystyle{plain}
\bibliography{publications}                        % 'publications' is the name of a BibTeX file

% Publications from a BibTeX file using the multibib package
%\section{Publications}
%\nocitebook{book1,book2}
%\bibliographystylebook{plain}
%\bibliographybook{publications}                   % 'publications' is the name of a BibTeX file
%\nocitemisc{misc1,misc2,misc3}
%\bibliographystylemisc{plain}
%\bibliographymisc{publications}                   % 'publications' is the name of a BibTeX file

%\clearpage
%-----       letter       ---------------------------------------------------------
% recipient data
%\recipient{Company Recruitment team}{Company, Inc.\\123 somestreet\\some city}
%\date{January 01, 1984}
%\opening{Dear Sir or Madam,}
%\closing{Yours faithfully,}
%\enclosure[Attached]{curriculum vit\ae{}}          % use an optional argument to use a string other than "Enclosure", or redefine \enclname
%\makelettertitle

%Lorem ipsum dolor sit amet, consectetur adipiscing elit. Duis ullamcorper neque sit amet lectus facilisis sed luctus nisl iaculis. Vivamus at neque arcu, sed tempor quam. Curabitur pharetra tincidunt tincidunt. Morbi volutpat feugiat mauris, quis tempor neque vehicula volutpat. Duis tristique justo vel massa fermentum accumsan. Mauris ante elit, feugiat vestibulum tempor eget, eleifend ac ipsum. Donec scelerisque lobortis ipsum eu vestibulum. Pellentesque vel massa at felis accumsan rhoncus.

%Suspendisse commodo, massa eu congue tincidunt, elit mauris pellentesque orci, cursus tempor odio nisl euismod augue. Aliquam adipiscing nibh ut odio sodales et pulvinar tortor laoreet. Mauris a accumsan ligula. Class aptent taciti sociosqu ad litora torquent per conubia nostra, per inceptos himenaeos. Suspendisse vulputate sem vehicula ipsum varius nec tempus dui dapibus. Phasellus et est urna, ut auctor erat. Sed tincidunt odio id odio aliquam mattis. Donec sapien nulla, feugiat eget adipiscing sit amet, lacinia ut dolor. Phasellus tincidunt, leo a fringilla consectetur, felis diam aliquam urna, vitae aliquet lectus orci nec velit. Vivamus dapibus varius blandit.

%Duis sit amet magna ante, at sodales diam. Aenean consectetur porta risus et sagittis. Ut interdum, enim varius pellentesque tincidunt, magna libero sodales tortor, ut fermentum nunc metus a ante. Vivamus odio leo, tincidunt eu luctus ut, sollicitudin sit amet metus. Nunc sed orci lectus. Ut sodales magna sed velit volutpat sit amet pulvinar diam venenatis.

%Albert Einstein discovered that $e=mc^2$ in 1905.

%\[ e=\lim_{n \to \infty} \left(1+\frac{1}{n}\right)^n \]

%\makeletterclosing

\end{document}
